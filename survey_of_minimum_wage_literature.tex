\documentclass[11pt]{amsart}
\usepackage{graphicx}
\usepackage{amsmath, amsfonts, amsthm, amssymb, latexsym, mathtools}
\usepackage{mathrsfs}
\usepackage{thmtools}
\usepackage{thm-restate}
\usepackage{cleveref}
\usepackage{etex, etoolbox, ifthen, url,csquotes}
\usepackage{kvoptions}
\usepackage{logreq}
\usepackage[american]{babel}
\usepackage[doublespacing]{setspace}
\usepackage[margin=1in]{geometry}
\usepackage[backend=biber, autopunct=true, authordate]{biblatex-chicago}
\usepackage{placeins}
\usepackage[page]{appendix}
\usepackage{siunitx}
\usepackage[T1]{fontenc}
\usepackage{csquotes}



\newtheorem{theorem}{Theorem}
\newtheorem{corollary}{Corollary}[theorem]
\newtheorem{lemma}{Lemma}
\pagestyle{plain}

\newcommand{\E}{\mathbb{E}}
\newcommand{\Var}{\mathbb{V}\mathrm{ar}}
\newcommand{\Cov}{\mathbb{C}\mathrm{ov}}
\newcommand{\N}{\mathcal{N}}
\newcommand{\lequals}{\stackrel{\mathcal{L}}{=}}
\newcommand{\aequals}{\stackrel{a.s.}{=}}
\newcommand{\I}{\mathbb{I}}
\newcommand{\F}{\mathcal{F}}
\newcommand{\qt}[1]{\lq\lq#1\rq\rq}


\title{Survey of Minimum Wage Literature}
\date{\today}
\author{Paul Sangrey}
\thanks{\textit{Email}: sangrey@sas.upenn.edu}
\addbibresource{survey_of_minimum_wage_literature.bib}



\begin{document}

\maketitle

One of the perennial questions in empirical economics is the effect on minimum wages on unemployment and wages.
In fact, the empirical analysis of this question goes back to one of the first empirical papers ever written \cite{obenauer1915effect}.
In \citeyear{card1994minimum}, \citeauthor{card1994minimum} present empirical evidence in their now seminal study that the elasticity is actually very close to zero, counter basic economic theory, which says it should be negative. 
(Although, no specific magnitude is implied.)
In so doing, they restarted the then stagnant literature analyzing this question, which has grown quite large in the intervening years.
This then raises the question what is the current state of literature?

This literature has yet to reach a consensus on the precise value of this elasticity.
Why might this be the case?
Two main issues dominate the literature.
The first is the optimal method for dealing with the spatial heterogeneity in employment growth.
To calculate the employment effect of minimum wage, you must estimate the employment level in the counterfactual case where the minimum wage was not changed then difference the two employment rates.
However, employment  trends in different directions at different rates in different regions, and so optimally forming this control is not trivial.
Consequently, various economists have proposed a large number of methods to do this.
The second major issue regards the difference between short-run and long-run elasticities.
One would a priori expect that the long-run elasticity is larger as companies have more time to adjust. 
However, the short-run elasticity is significantly easier to measure, and so most empirical papers have assumed that the two elasticities are equal.


\section*{Empirical Estimates of the Elasticity}


Perhaps the most useful article to examine when considering the current state of the literature is \cite{neumark2014revisiting}.
The authors of this article survey the literature and provide their own empirical estimates using a synthetic control approach. 
The literature has dealt with the spatial heterogeneity at level of its unit of analysis, the country, by constructing comparison counties using different methods.
The various papers also usually include various spatial (such as state-level) and time dummy variables as well as other relevant regressors, and then group the counties in different ways.
\citeauthor{neumark2014revisiting} use a synthetic control approach through which they let the data identify these groups.
In doing so, they find that their is a teen disemployment elasticity effect of minimum wages is about $-0.15$. 

However, the two main articles that they discuss \cite{allegretto2011minimum} and \cite{dube2010minimum} use different methods for dealing with heterogeneity and find approximately zero effects. 
\cite{dube2010minimum} compare contiguous cross-border country pairs. 
They consider the cross-border pairs across the entire country between 1990 and 2006 where there was a difference in the minimum wages. 
\cite{allegretto2011minimum}, on the other hand, group the states by census tract.
In addition, they also argue that long-run effects are very small in magnitude using a distributed lag approach to pick up both pre-existing trends and differentiate between long-run and short-run effects. 

In another interesting vein of research, several authors have considered the effect on the recent real declines in minimum wages.
\cite{bosch2010minimum} finds that almost entirely of the growth in inequality at the bottom end of the earnings inequality in Mexico can be attributed to this.
\cite{autor2016contribution} arrive at somewhat similar conclusions arguing that a substantial amount, but by no means all, of the increased inequality since 1980 between the 10th and 50th percentiles can be explained this way.
This is particularly interesting because the minimum wage typically binds significantly below the 10th percentile.

In another related strain of literature, \cite{neumark2012effects} finds that living wage laws reduce employment among the populations that they are trying to help.
\cite{addison2012effect}, on the other hand, finds that the elasticity of low-wage employment is approximately zero, and instead employment appears to exhibit an independent downward trend in regions where minimum wages are enacted. 


\section*{Theoretical work concerning the elasticity}

Two main sets of papers are of interest. 
In \cite{sorkin2015are}, \citeauthor{sorkin2015are} builds a dynamic industry equilibrium model that reconciles the small estimated short-run employment effects with much larger long-run employment effects. 
He also makes the interesting point that because most minimum wages in the United States have not been indexed to inflation, the increases themselves are necessarily short-run. 
That is, the long-run will simply never come.  
In \cite{aaronson2016industry}, the authors show that restaurant exit and entry both rise following a hike, and that the rise in entry is concentrated in chains which tend to be more capital-intensive. 
On the other hand, non-entrants see no significant employment changes. 
They also build a theoretical model to fit these facts.

The other interesting theoretical work is an optimality result by \citeauthor{lee2012optimal} who show that if a government values redistribution towards low wage workers, then in a perfectly competitive labor market a binding minimum wage is desirable. 
Surprisingly, this result holds true in the presence of optimal nonlinear taxes and transfers. 
The minimum wage enhances the effectiveness of the transfers through incidence effects as the assume that the unemployment hits the lowest surplus workers first. 

\section*{Conclusion}

As a consequence of the spatial heterogeneity in employment growth, creating good estimates of a particular county's employment rate among low-wage workers in the absence of a change in the minimum wage is quite difficult. 
Hence, reliably and precisely estimating the relevant elasticity still has not been done, but this elasticity is likely rather small. 
These difficulties are even stronger if one is considering long-run elasticities because the environments have had more time to change. 
In addition, most minimum wage hikes in the U.S. have only been temporary as a consequence of inflation.
We do have, however, strong reason to believe that the long-run elasticity is likely larger in magnitude than the short-run elasticity. 

\newpage
\printbibliography



\end{document}


