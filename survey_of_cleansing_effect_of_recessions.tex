\documentclass[11pt]{amsart}
\usepackage{graphicx}
\usepackage{amsmath, amsfonts, amsthm, amssymb, latexsym, mathtools}
\usepackage{mathrsfs}
\usepackage{thmtools}
\usepackage{thm-restate}
\usepackage{cleveref}
\usepackage{etex, etoolbox, ifthen, url,csquotes}
\usepackage{kvoptions}
\usepackage{logreq}
\usepackage[american]{babel}
\usepackage[doublespacing]{setspace}
\usepackage[margin=1in]{geometry}
\usepackage[backend=biber, autopunct=true, authordate]{biblatex-chicago}
\usepackage{placeins}
\usepackage[page]{appendix}
\usepackage{siunitx}
\usepackage[T1]{fontenc}
\usepackage{csquotes}
%\usepackage{titling}%


\newtheorem{theorem}{Theorem}
\newtheorem{corollary}{Corollary}[theorem]
\newtheorem{lemma}{Lemma}
\pagestyle{plain}

\newcommand{\E}{\mathbb{E}}
\newcommand{\Var}{\mathbb{V}\mathrm{ar}}
\newcommand{\Cov}{\mathbb{C}\mathrm{ov}}
\newcommand{\N}{\mathcal{N}}
\newcommand{\lequals}{\stackrel{\mathcal{L}}{=}}
\newcommand{\aequals}{\stackrel{a.s.}{=}}
\newcommand{\I}{\mathbb{I}}
\newcommand{\F}{\mathcal{F}}
\newcommand{\qt}[1]{\lq\lq#1\rq\rq}


\thanks{\textit{Email}: sangrey@sas.upenn.edu}
\date{\today}
\title{Survey of Cleansing Effect of Recessions Literature}
\author{Paul Sangrey}		
\addbibresource{cleansing_effect_of_recessions_references.bib}

\begin{document}

\maketitle 

At least since \cite{schumpeter1939business}, economists have debated the effect of business cycles on economic growth. 
In 1994, the theoretical literature was significantly furthered by \citeauthor{caballero1994cleansing} in their seminal paper \citetitle{caballero1994cleansing} which created a model where lower productivity firms are more likely to be destroyed in recessions. 
As a consequence, recessions increase average productivity by \textit{cleansing} the economy of unproductive firms. 
On the other hand, various authors have argued that increased frictions in the form of credit tightening during recessions may reduce the reallocation of productive resources towards their most effective use. 
In addition, certain forms of heterogeneity when combined with search frictions can also form a \textit{sullying} effect of recessions as in \cite{barlevy2002sullying}. 
Consequently, sclerosis may actually increase during recessions instead of decrease.
Since all of these effects likely happen, it is an empirical question which one dominates in practice.


In addition to the theoretical issues, evidence suggests that cumulative restructuring is lower following recessions and 
To deal with these issues \citeauthor{caballero2005cost} wrote another paper \citetitle{caballero2005cost} in which they build a model where heterogeneity in firm productivity and financial constraints causes reduced reallocation. 
Their main conclusion is that recessions lead to less total restructuring and that this is costly. 

To fast-forward to the present, in the aftermath of the Great Recession and its slow recovery renewed interest has arisen in this topic.
Haltiwanger and various coauthors have written several articles on related topics using plant-level data.
They focus primarily on manufacturing firms, as do most other papers in this literature, as managing physical productivity in other firms is quite difficult.  
\cite{davis2012labor} shows that job destruction and layoffs move together over time while quits move counter to both. 
It also finds that plotting hiring and separation as function of establishment-level growth rates exhibits a nonlinear \qt{hockey-stick} behavior where the relation is nearly flat to a point and then rises rapidly afterwards. 
A more recent paper, \cite{foster2016reallocation}, mentions that prior research suggests that the behavior of macroeconomic variables during recession in the early 1980s is consistent with the cleansing hypothesis. 
However, responsiveness of job creation and destruction to cyclical contractions changed dramatically during the Great Recession. 
As before, job destruction rose significantly, but, unlike previous ones, job creation fell significantly as well.
(In previous recessions job creation only fell mildly.)
In addition, it finds that the marginal impact of productivity on exit fell significantly during the Great Recession although it remained nonzero. 

Another paper by Haltwianger and coauthors, \cite{fort2013how} shows that young businesses have very different cyclical dynamics than older businesses. 
It further argues that young and small businesses are particularly sensitive to housing prices and hence were hit atypically hard by the Great Recession.
It also mentions that the recent literature (such as \cite{gertler1994monetary} and \cite{chari2007gertler}) has not reached a consensus on whether, in general, large firms or small firms have a larger response to recessions.
Rather, which type of firm appears more responsive depends upon the economic indicators considered.

In another recent paper, \citeauthor{kehrig2011cyclicality} applies a \cite{olley1996dynamics} methodology to plant-level data in his paper \citetitle{kehrig2011cyclicality}.
He shows that cross-sectional dispersion of total factor productivity levels in U.S. manufacturing is countercyclical being about 10\% more spread out in recessions. 
This is primarily driven by a higher share of unproductive firms in recessions. 
In addition, this pattern is more pronounced in durable goods industries. 
He then builds along the lines of \cite{ghironi2005international} to explain these empirical regularities. 

A more theoretical paper, \cite{ouyang2009scarring} develops a model where recessions impede the development of potentially superior firms by destroying them in infancy while still allowing for the cleansing effects in the style of \cite{caballero1994cleansing}. 
It calibrates this model using statistics on entry, exit and productivity differentials and finds that the scarring effect dominates leading to lower average productivity in recessions. 

\cite{bloom2009impact}, on the other hand, is mainly concerned with the the effect of uncertainty on economic growth in a macroeconomic context with stochastic volatility.
However, in considering that question, it shows that cross-sectional revenue productivity dispersion is countercyclical at both the plant and firm level and that this dispersion is strongly correlated with productivity growth and with stock market volatility.  
It then builds a theoretical model where an uncertainty shock causes firms to temporarily freeze their hiring and firing and hence slows productivity growth by freezing reallocation of workers towards more productive units. 


To wrap up, productivity dispersion increases during recessions, but it is not clear whether productivity-enhancing reallocation dominates the various effects that are detrimental to productivity.
This lack of consensus holds both both theoretically and empirically.
It is likely, however, that the Great Recession had a detrimental effect on productivity and the positive reallocation of resources was seems to be rather muted, which may help to explain the sluggish growth in its aftermath.
This may well be caused by a particularly strong effect on young firms during that recessions, which itself may be related to the sharp fall in housing prices. 



\newpage
\printbibliography

\end{document}


